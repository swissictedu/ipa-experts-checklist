\section{Bewertung}
Die Bewertung muss unmittelbar im Anschluss an die Präsentation stattfinden. An der Sitzung
nehmen nur noch die verantwortliche Fachkraft, die Experten, sowie der fachkompetente Gast teil. Das Ziel ist eine Note zu finden, mit welcher du und die verantwortliche Fachkraft zufrieden sind. Du bist der Sitzungsleiter. In der Regel wird die Bewertung eine gute Stunde dauern.

\begin{taskitem}{Gesamteindruck}
  Frage die verantwortliche Fachkraft zuerst nach dem Gesamteindruck.
\end{taskitem}
\begin{taskitem}{Rechtzeitige Abgabe des Berichts}
  Stelle fest, ob der Bericht rechtzeitig hochgeladen wurde. Ansonsten gibt es einen Notenabzug nach Vorgaben der Prüfungsleitung.
\end{taskitem}
\begin{taskitem}{Bewertung der verantwortlichen Fachkraft}
  Diskutiere kurz die Bewertung mit der verantwortlichen Fachkraft und vergleichen sie mit deren Notizen.
\end{taskitem}
\begin{taskitem}{Bewertung vergleichen}
  Anschliessend besprichst du die Bewertung der verantwortlichen Fachkraft und diskutierst diese mit ihm und dem Nebenexperten. Verschwende keine Zeit, wenn beide Bewertungsvorschläge für ein Kriterium gleich sind. Vermeide Grundsatzdiskussionen und ausschweifendes Fachsimpeln.
\end{taskitem}
\begin{taskitem}{Bewertungsraster auf PkOrg ergänzen}
  Fülle das Bewertungsraster in PkOrg aus. Beachte, dass für jedes Kriterium eine nachvollziehbare Begründung eingetragen werden muss. (Bsp. \enquote{unvollständig}: was fehlt?) Denke daran: die Punkte repräsentieren keine Note, sondern eine Gütestufe des Kriteriums. Erst die Summe aller Punkte ergibt einen Notenvorschlag.
\end{taskitem}
\begin{taskitem}{Fachgespräch auswerten}
  Die effektiv geführten Dialoge des Fachgesprächs sind zu gruppieren und zu den vorgesehenen Fachgesprächskriterien zuzuordnen. Spontane Fragen bedingen unter Umständen eine Anpassung der vorbereiteten Kriterien.
\end{taskitem}
\begin{taskitem}{Einigkeit}
  Der Notenvorschlag erscheint im PkOrg sobald alle Gütestufen und Bemerkungen eingegeben sind - entspricht das Resultat den Erwartungen? Bei Uneinigkeit können in PkOrg das Häkchen setzen und der Notenkonferenz den Entscheid überlassen. Bitte diese Möglichkeit nur benutzen, wenn es wirklich nicht anders geht.
\end{taskitem}
\begin{taskitemwithoutcomment}{Signatur der Bewertung}
  Die Bewertung muss von der verantwortlichen Fachkraft nochmals signiert werden.
\end{taskitemwithoutcomment}
\begin{taskitemwithoutcomment}{Unterlagen und Notizen einsammeln}
  Alles Material, Notizen der verantwortlichen Fachkraft und des Nebenexperten einsammeln und mitnehmen.
\end{taskitemwithoutcomment}
\begin{taskitemwithoutcomment}{Notenvorschlag}
  Mache die verantwortliche Fachkraft darauf aufmerksam, dass er den Notenvorschlag dem Kandidaten nicht mitteilen darf, weil dieser auch im Nachhinein (beim Quervergleich) durch die Prüfungsleitung verändert werden kann.
\end{taskitemwithoutcomment}
\begin{taskitemwithoutcomment}{Vertraulichkeit}
  Mache die verantwortliche Fachkraft darauf aufmerksam, dass alle Unterlagen vertraulich behandelt werden und im Kreis der Experten verbleiben.
\end{taskitemwithoutcomment}
\begin{taskitemwithoutcomment}{Danksagung}
  Bedanke dich bei der verantwortlichen Fachkraft für ihren Einsatz und ermuntere sie, ebenfalls als Experte mitzuwirken. Weitere Informationen dazu können auf \href{https://pk19.ch}{der Webseite der Prüfungsorganisation} gefunden werden.
\end{taskitemwithoutcomment}
\begin{taskitem}{Austausch mit Nebenexperte}
  Nach Verabschiedung der verantwortlichen Fachkraft nutze die Gelegenheit und besprechen den Ablauf des Prüfungstages kritisch. Beispielfragen: \enquote{Wie war das Verhältnis zum Kandidaten? Wie war der Führungsstil? Gab es Situationen, in der der Kandidat sich unwohl fühlte? War das Fachgespräch flüssig und angemessen? Wo hätte man mehr Aufmerksamkeit schenken können? Strenger/weniger streng reagieren?}
\end{taskitem}
