\section{Zwischenfälle: Pannen, Krankheit, Knappheit, ...}
Bei der Präsentation, dem Fachgespräch und der Bewertung ist normalerweise ein zweiter Experte als Nebenexperte dabei. Er sorgt für vollständige Notizen und beteiligt sich am Fachgespräch und der Bewertung. Zwischenfälle müssen vom Kandidaten oder von der verantwortlichen Fachkraft umgehend dem Hauptexperten gemeldet werden.

Es ist wichtig, dass alle Zwischenfälle lückenlos über die History und ggf. Anhänge dokumentiert sind.

\begin{enumerate}
  \item Kläre die Umstände möglichst genau ab und finde denkbare Massnahmen.
  \begin{itemize}
    \item Verschieben des Abgabetermins
    \item Verlängern der IPA (1 bis 3 Tage max.)
    \item Ändern der Aufgabenstellung. Achtung: schriftlich festhalten und ggf. die Bewertungskriterien anpassen.
    \item Verschieben der ganzen Facharbeit
    \item Abbrechen und später eine neue Facharbeit durchführen
  \end{itemize}
  \item Sprich dich mit dem jeweiligen Chefexperten ab.
  \item Notiere den Entscheid in der PkOrg-History. Trage vereinbarte Termine (Wiederaufnahme der Arbeit, neuer Abgabetermin, ...) ein. Mit der PkOrg-Funktion \enquote{Hinweis an Kandidat} erreichst du alle Beteiligten.
  Informiere ggf. die Prüfungsleitung, damit die Durchführungstermine auf PkOrg
  angepasst werden können. Führe die Besuchstermine auf PkOrg nach, speziell den Präsentationstermin.
  \item Für Krankheit und Unfall ist ab dem ersten ausfallenden Tag ein Arztzeugnis nötig (Ablage auf PkOrg im Dokumentenpool).
  \item Sollte sich zeigen, dass Bewertungskriterien ohne Verschulden des Kandidaten nicht erfüllbar sind, nehmen Sie mit der Prüfungsleitung Kontakt auf.
\end{enumerate}

Falls du das Gefühl hast, dass die Facharbeit ungenügend abgeschlossen werden könnte, solltest ...
\begin{itemize}
  \item ... über PkOrg einen Nebenexperten verlangen, falls nicht schon einer zugeteilt ist.
  \item ... weitere Fachkräfte um Rat fragen.
  \item ... in besonders heiklen Fällen beim Chefexperten oder bei der IPA-Prüfungsleitung eine Zweitbeurteilung verlangen.
  \item ... noch genauere Notizen machen (Einsprache, Rekurs). Es muss für Dritte ersichtlich sein, was erreicht wurde oder was fehlt, um die entsprechende Punktvergabe zu rechtfertigen.
  \item ... Aussagen im IPA-Bericht eventuell zusätzlich verifizieren.
\end{itemize}

Unpünktlichkeit oder gar Fernbleiben macht einen denkbar schlechten Eindruck. Deshalb ist es äussert wichtig, alle Beteiligten und bei Bedarf die Prüfungsleitung frühzeitig oder bei Pannen umgehend zu informieren und Ersatzlösungen zu organisieren.